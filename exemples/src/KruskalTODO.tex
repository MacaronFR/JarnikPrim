\documentclass[a4paper,12pt]{article}

\usepackage[french]{babel}
\usepackage{fancyhdr,lastpage}
\usepackage{pstricks-add}
\usepackage{tikz}
\usepackage[french,ruled,linesnumbered,vlined]{algorithm2e}
\usepackage{multicol}

\pagestyle{fancy}
\fancyhead[L]{Démonstration Jàrnik Prim}
\fancyhead[R]{MacaronFR/TheorieGraphes}
\fancyfoot[C]{Page \thepage{} / \pageref{LastPage}}

\begin{document}

\paragraph{Présentation}
L'algorithme de Jàrnik-Prim sert à trouver l'arbre recouvrant minimum dans un graphe non orienté pondéré.
Il part d'un sommet du graphe, puis va relier tout les sommet n'appartenant à l'arbre de résultat dont l'arrête à le poids minimum.

\paragraph{Algorithme}\hfill
\begin{algorithm}
\caption{Kruskal}
$G$ un graphe connexe\\
choisir un sommet $s$ de G\\
$A \gets \{s, o\}$\;
\Tq{A n'est pas un sous-graphes recouvrant de G}{
	parmis les arrêtes $xy$ de $G$ avec $x$ dans $A$ et $y$ dans $G - A$\\
	en choisir une de couts minimum\\
	ajouter à A le sommet $y$ et l'arrête $xy$
}
retourner $A$
\end{algorithm}

\paragraph{Trace}
Les sommets et les arrêtes de A seront noté en rouge\\

\begin{multicols}{2}
\begin{tikzpicture}
\matrix[nodes={draw,circle}, row sep=0.5cm, column sep=0.7cm]{
	;&;&\node[fill=red](a){A}; \\
	;&\node(b){B}; &;&;&\node(c){C};\\
	;&;&;\node(e){E};\\
	\node(d){D};&;&;&;&\node(f){F};\\
	;&;&\node(g){G};\\
};
\tikzstyle{fleche} = [draw,thick,>=latex]
\draw[fleche] (a) -- node[above left]{$5$}(b);
\draw[fleche] (a) -- node[above right]{$4$}(c);
\draw[fleche] (b) -- node[above]{$10$}(c);
\draw[fleche] (b) -- node[above left]{$2$}(d);
\draw[fleche] (b) -- node[above right]{$2$}(e);
\draw[fleche] (c) -- node[above left]{$8$}(e);
\draw[fleche] (c) -- node[right]{$5$}(f);
\draw[fleche] (d) -- node[above left]{$1$}(e);
\draw[fleche] (d) -- node[above right]{$6$}(g);
\draw[fleche] (e) -- node[above right]{$3$}(f);
\draw[fleche] (f) -- node[above left]{$6$}(g);
\end{tikzpicture}

\begin{tikzpicture}
\matrix[nodes={draw,circle}, row sep=0.5cm, column sep=0.7cm]{
	;&;&\node[fill=red](a){A}; \\
	;&\node(b){B}; &;&;&\node[fill=red](c){C};\\
	;&;&;\node(e){E};\\
	\node(d){D};&;&;&;&\node(f){F};\\
	;&;&\node(g){G};\\
};
\tikzstyle{fleche} = [draw,thick,>=latex]
\draw[fleche] (a) -- node[above left]{$5$}(b);
\draw[fleche,red] (a) -- node[above right]{$4$}(c);
\draw[fleche] (b) -- node[above]{$10$}(c);
\draw[fleche] (b) -- node[above left]{$2$}(d);
\draw[fleche] (b) -- node[above right]{$2$}(e);
\draw[fleche] (c) -- node[above left]{$8$}(e);
\draw[fleche] (c) -- node[right]{$5$}(f);
\draw[fleche] (d) -- node[above left]{$1$}(e);
\draw[fleche] (d) -- node[above right]{$6$}(g);
\draw[fleche] (e) -- node[above right]{$3$}(f);
\draw[fleche] (f) -- node[above left]{$6$}(g);
\end{tikzpicture}
\end{multicols}
\newpage
\begin{multicols}{2}
\begin{tikzpicture}
\matrix[nodes={draw,circle}, row sep=0.5cm, column sep=0.7cm]{
	;&;&\node[fill=red](a){A}; \\
	;&\node[fill=red](b){B}; &;&;&\node[fill=red](c){C};\\
	;&;&;\node(e){E};\\
	\node(d){D};&;&;&;&\node(f){F};\\
	;&;&\node(g){G};\\
};
\tikzstyle{fleche} = [draw,thick,>=latex]
\draw[fleche,red] (a) -- node[above left]{$5$}(b);
\draw[fleche,red] (a) -- node[above right]{$4$}(c);
\draw[fleche] (b) -- node[above]{$10$}(c);
\draw[fleche] (b) -- node[above left]{$2$}(d);
\draw[fleche] (b) -- node[above right]{$2$}(e);
\draw[fleche] (c) -- node[above left]{$8$}(e);
\draw[fleche] (c) -- node[right]{$5$}(f);
\draw[fleche] (d) -- node[above left]{$1$}(e);
\draw[fleche] (d) -- node[above right]{$6$}(g);
\draw[fleche] (e) -- node[above right]{$3$}(f);
\draw[fleche] (f) -- node[above left]{$6$}(g);
\end{tikzpicture}

\begin{tikzpicture}
\matrix[nodes={draw,circle}, row sep=0.5cm, column sep=0.7cm]{
	;&;&\node[fill=red](a){A}; \\
	;&\node[fill=red](b){B}; &;&;&\node[fill=red](c){C};\\
	;&;&;\node(e){E};\\
	\node[fill=red](d){D};&;&;&;&\node(f){F};\\
	;&;&\node(g){G};\\
};
\tikzstyle{fleche} = [draw,thick,>=latex]
\draw[fleche,red] (a) -- node[above left]{$5$}(b);
\draw[fleche,red] (a) -- node[above right]{$4$}(c);
\draw[fleche] (b) -- node[above]{$10$}(c);
\draw[fleche,red] (b) -- node[above left]{$2$}(d);
\draw[fleche] (b) -- node[above right]{$2$}(e);
\draw[fleche] (c) -- node[above left]{$8$}(e);
\draw[fleche] (c) -- node[right]{$5$}(f);
\draw[fleche] (d) -- node[above left]{$1$}(e);
\draw[fleche] (d) -- node[above right]{$6$}(g);
\draw[fleche] (e) -- node[above right]{$3$}(f);
\draw[fleche] (f) -- node[above left]{$6$}(g);
\end{tikzpicture}
\end{multicols}

\begin{multicols}{2}
\begin{tikzpicture}
\matrix[nodes={draw,circle}, row sep=0.5cm, column sep=0.7cm]{
	;&;&\node[fill=red](a){A}; \\
	;&\node[fill=red](b){B}; &;&;&\node[fill=red](c){C};\\
	;&;&;\node[fill=red](e){E};\\
	\node[fill=red](d){D};&;&;&;&\node(f){F};\\
	;&;&\node(g){G};\\
};
\tikzstyle{fleche} = [draw,thick,>=latex]
\draw[fleche,red] (a) -- node[above left]{$5$}(b);
\draw[fleche,red] (a) -- node[above right]{$4$}(c);
\draw[fleche] (b) -- node[above]{$10$}(c);
\draw[fleche,red] (b) -- node[above left]{$2$}(d);
\draw[fleche] (b) -- node[above right]{$2$}(e);
\draw[fleche] (c) -- node[above left]{$8$}(e);
\draw[fleche] (c) -- node[right]{$5$}(f);
\draw[fleche,red] (d) -- node[above left]{$1$}(e);
\draw[fleche] (d) -- node[above right]{$6$}(g);
\draw[fleche] (e) -- node[above right]{$3$}(f);
\draw[fleche] (f) -- node[above left]{$6$}(g);
\end{tikzpicture}

\begin{tikzpicture}
\matrix[nodes={draw,circle}, row sep=0.5cm, column sep=0.7cm]{
	;&;&\node[fill=red](a){A}; \\
	;&\node[fill=red](b){B}; &;&;&\node[fill=red](c){C};\\
	;&;&;\node[fill=red](e){E};\\
	\node[fill=red](d){D};&;&;&;&\node[fill=red](f){F};\\
	;&;&\node(g){G};\\
};
\tikzstyle{fleche} = [draw,thick,>=latex]
\draw[fleche,red] (a) -- node[above left]{$5$}(b);
\draw[fleche,red] (a) -- node[above right]{$4$}(c);
\draw[fleche] (b) -- node[above]{$10$}(c);
\draw[fleche,red] (b) -- node[above left]{$2$}(d);
\draw[fleche] (b) -- node[above right]{$2$}(e);
\draw[fleche] (c) -- node[above left]{$8$}(e);
\draw[fleche] (c) -- node[right]{$5$}(f);
\draw[fleche,red] (d) -- node[above left]{$1$}(e);
\draw[fleche] (d) -- node[above right]{$6$}(g);
\draw[fleche,red] (e) -- node[above right]{$3$}(f);
\draw[fleche] (f) -- node[above left]{$6$}(g);
\end{tikzpicture}
\end{multicols}

\begin{multicols}{2}
\begin{tikzpicture}
\matrix[nodes={draw,circle}, row sep=0.5cm, column sep=0.7cm]{
	;&;&\node[fill=red](a){A}; \\
	;&\node[fill=red](b){B}; &;&;&\node[fill=red](c){C};\\
	;&;&;\node[fill=red](e){E};\\
	\node[fill=red](d){D};&;&;&;&\node[fill=red](f){F};\\
	;&;&\node[fill=red](g){G};\\
};
\tikzstyle{fleche} = [draw,thick,>=latex]
\draw[fleche,red] (a) -- node[above left]{$5$}(b);
\draw[fleche,red] (a) -- node[above right]{$4$}(c);
\draw[fleche] (b) -- node[above]{$10$}(c);
\draw[fleche,red] (b) -- node[above left]{$2$}(d);
\draw[fleche] (b) -- node[above right]{$2$}(e);
\draw[fleche] (c) -- node[above left]{$8$}(e);
\draw[fleche] (c) -- node[right]{$5$}(f);
\draw[fleche,red] (d) -- node[above left]{$1$}(e);
\draw[fleche,red] (d) -- node[above right]{$6$}(g);
\draw[fleche,red] (e) -- node[above right]{$3$}(f);
\draw[fleche] (f) -- node[above left]{$6$}(g);
\end{tikzpicture}
\end{multicols}

\end{document}
